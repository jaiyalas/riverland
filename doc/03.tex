% !TEX TS-program = xelatex
% \documentclass[a4paper]{article}
\documentclass[a4paper,twocolumn]{article}

\usepackage{bussproofs}
\usepackage{amssymb}
\usepackage[colorlinks=true]{hyperref}

\title{Linear Type from Linear Logic}
\author{Jen-Shin Lin}

\begin{document}

\maketitle

\section{linear programming}

   \cite{Wadler93}
      A Taste of Linear Logic
   \cite{Benton92}
      Term assignment for intuitionistic linear logic
   \cite{Benton93}
      A Term Calculus for Intuitionistic Linear Logic
   \cite{Girard95}
      Linear Logic: Its Syntax And Semantics
   \cite{Ronchi94}
      Lambda Calculus and Intuitionistic Linear Logic
   \cite{Turner98}
      Operational Interpretations of Linear Logic
   \cite{Novitzka07}
      Resource-oriented Programming Based on Linear Logic

\section{reversible computation}

\subsection{reversible communication process}

   \cite{Wadler12}
      Propositions As Sessions
   \cite{Brown16}
      Reversible Communicating Processes

\subsection{reversible language}

   \cite{Yokoyama11}
      Towards a Reversible Functional Language
   \cite{Yokoyama16}
      Fundamentals of Reversible Flowchart Languages
   \cite{James12}
      Information Effects
   \cite{James13}
      Isomorphic Interpreters from Logically Reversible Abstract Machines
   \cite{James14}
      Theseus: A high level language for reversible computing
   \cite{Sparks14}
      Superstructural Reversible Logic



\section{\_old}


Writing programs with a language which supports linear operations has some fascinating benefits. However, providing good properties means the presence of powerful syntax. And powerful syntax equals to complicate syntax that unavoidably leads to the increacment of programming difficulty for human. Therefore one naturally will want to apply type system to a linear language as an expilict guarding mechanism. Since the linear logic provides an operational semantics cite{Girard95} for using and managing resources, it's natural to apply the well-known Curry–Howard correspondence and construct a linear type system from linear logic.

\begin{prooftree}
\AxiomC{$\Gamma \vdash A$}
\AxiomC{$B, \Delta \vdash C$}
\RightLabel{($\multimap$-L)}
\BinaryInfC{$\Gamma, A \multimap B, \Delta \vdash C$}
\end{prooftree}
\begin{prooftree}
\AxiomC{$\Gamma, A \vdash B$}
\RightLabel{($\multimap$-R)}
\UnaryInfC{$\Gamma \vdash A \multimap B$}
\end{prooftree}

If we force to produce typing rules from L/R rules, things will become somewhat an awkward situiation. First of all, the term assignment for this kind of typing rules will be more complicated. This will immediately make those tpying rules become unsuitable for applying in safety proving.

One of the alternative choices would be translating L/R rules into introduction/elimination rules. Namely rewrite linear logic's deductive rules as in natural deduction system. The following rules are the I/E rules corresponding to those in above example. As it presets itself, with I/E rules one can easily construct and destruct terms from or back to its premises.

\begin{prooftree}
\AxiomC{$\Gamma \vdash A \multimap B$}
\AxiomC{$\Delta \vdash A$}
\RightLabel{($\multimap$-E)}
\BinaryInfC{$\Gamma, \Delta \vdash B$}
\end{prooftree}
\begin{prooftree}
\AxiomC{$\Gamma, A \vdash B$}
\RightLabel{($\multimap$-I)}
\UnaryInfC{$\Gamma \vdash A \multimap B$}
\end{prooftree}


In the design of our linear language, as we mentioned in the last report, the program will be allowed to be evaluated within linear mode (as default) or non-linear mode. This is done by providing a syntax-level function, $\mathtt{bang}$, that can be used to annotate a variable or expression for being treated as a non-linear resource. To be able to evaluate program in both modes, the way to go is to seperate environment context into to parts: linear context and non-linear context. Nevertheless they cannot be entirely seperated because of the fact that they are both a part of a big abstract context in which we allow no name duplicate. To solve this problem we introduce the dual distinguished-domain, which is defined as $\mathtt{DualDomDist}$ in file: \textit{Ctx.agda}\footnotemark[1]\footnotetext[1]{links to source codes are listed in the last paragraph of this article}. This dual domain design can make proving easier and clearer, however, it also generate lots of requirements of proofs of related properties, which can be found in file: \textit{Auxiliaries.agda}\footnotemark[1].

\bibliographystyle{acm}
\bibliography{ref}

\end{document}
